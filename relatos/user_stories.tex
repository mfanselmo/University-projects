\documentclass[12pt, letterpaper, notitlepage]{article}
\usepackage{geometry}
\usepackage[utf8]{inputenc}
\usepackage[spanish]{babel}
\usepackage{amsmath}
\usepackage{amsfonts}
\usepackage{amssymb}
\usepackage{dirtytalk}

\geometry{
	letterpaper,
	left=30mm,
	right=30mm,
	top=30mm,
	bottom=30mm
}


\newcommand{\ramo}{Ingeniería de Software}
\newcommand{\sigla}{ IIC2143-2}

\usepackage{fancyhdr}
\usepackage{fancyvrb}
\usepackage{subcaption}



\pagestyle{fancy}
\fancyhf{}
\renewcommand{\headrulewidth}{0.35pt}
\renewcommand{\footrulewidth}{0.35pt}
\lhead{\footnotesize Pontificia Universidad Católica de Chile}
\rhead{\footnotesize \ramo \sigla}
\cfoot{\thepage}


\usepackage{multicol}

\title{\textbf{Entrega 0}}
\author{Martín Anselmo, Ignacio Contreras, Rodrigo Hanuch}
%\date{\today}

\begin{document}
	
	\clearpage\maketitle
	\thispagestyle{empty}
	
	\newpage
	
	\section*{Relatos de usuario}
	\newcommand{\story}[3]{Como {#1}, quiero {#2} para {#3}.}
	
	\begin{enumerate}
		
		\item \story{usuario}{poder registrarme en la aplicacion}	
		{así poder guardar mis foros preferidos y otras preferencias.}
		
		
		\item \story{usuario}{recibir un correo electrónico cuando cree una cuenta}	
		{así tener una confirmación del proceso}
		\begin{enumerate}
			\item Validación de correo electrónico (que exista)
			\item Link de confirmación en el correo		
		\end{enumerate}
		
		\item \story{usuario}{registrarme con mi correo electrónico}{para así no tener que 
			dar información personal}
		\begin{enumerate}
			\item Página de registro
			\item Verificar contraseña (escribir dos veces)
		\end{enumerate}
		
		
		\item \story{usuario}{poder cambiar informacion de mi usuario}{así arreglarla en caso de que algo cambie.}
		\begin{enumerate}
			\item Página o pestaña de configuración de cuenta
			\item Correo electrónico de confirmación
		\end{enumerate}
		
		
		\item \story{usuario}{poder hacer publicaciones}{así poder mostrarle cosas a otros usuarios o invitados}
		\begin{enumerate}
			\item Dentro de la página del foro debe haber un cuadro u opción \say{Crear post} o similiar
		\end{enumerate}
		
		
		\item \story{usuario}{poder comentar publicaciones en los foros}{asi dar mi opinión en algún post, o responderle a otro usuario}
		\begin{enumerate}
			\item Dentro de la página del foto debe haber un cuadro u opción de \say{Agregar comentario} o similar
		\end{enumerate}
		
		
		\item \story{usuario}{poder votar en comentarios y publicaciones}{asi expresar lo que pienso de manera simple en publicaciones y comentarios}
		\begin{enumerate}
			\item Comentarios de posts deben tener \say{Upvote} o \say{Downvote} para expresar el voto propio si se quiere
		\end{enumerate}
		
		
		\item \story{usuario}{entrar a la página web con mi correo electrónico}{para poder recordar
			fácilmente mi \say{log-in}}
		\begin{enumerate}
			\item Página de Log-In
			\item Comprobación de usuario y contraseña correcta
		\end{enumerate}
		
		\item \story{usuario}{que me lleguen notificaciones cuando esté en la página web}		
		{así poder saber que está pasando en los foros suscritos o comentados}
		\begin{enumerate}
			\item Sección de notificaciones en la página
			\item Al interactuar con una publicación mandar cambios al servidor
			\item Servidor manda aviso al usuario
		\end{enumerate}
		
		
		
		
		\item \story{usuario}{poder guardar y eliminar publicaciones en mis favoritos}{asi verlas mas facilmente si quiero volver a ellas.}
		\begin{enumerate}
			\item Dentro de perfil presentar una lista de favoritos (posiblemente en forma de scroll) junto con la información de la cuenta
			\item Poder eliminar una publicación desde la lista y agregar a la lista desde los posts
		\end{enumerate}
		
		\item \story{usuario}{poder editar y eliminar comentarios realizados}{borrar las que no me gusten o emjorar respuestas.}
		\begin{enumerate}
			\item Dentro de perfil presentar una lista de favoritos (posiblemente en forma de scroll) junto con la información de la cuenta
			\item Poder eliminar una publicación desde la lista y agregar a la lista desde los posts
		\end{enumerate}
		
		
		\item \story{usuario}{poder ver un ranking de foros con mas suscriptores}{asi saber que esta de moda en el momento}
		\begin{enumerate}
			\item Implementar una lista de foros y poder filtrar por cantidad de subscriptores
			\item Foros deben tener un número de subscriptores que se actualice en la base de datos con cada subscripción o desubscripción
		\end{enumerate}
		
		
		\item \story{usuario}{poder ver un ranking de usuarios con mas reputacion}{asi saber quienes tienen las mejores publicaciones y verlas.}
		\begin{enumerate}
			\item Sección de usuarios y poder filtrarlos por reputación y ordenarlos
		\end{enumerate}
		
		
		\item \story{usuario}{poder buscar foros por nombre}{encontrar de manera sencilla foros con mis gustos}
		\begin{enumerate}
			\item Barra de búsqueda de foros por nombres exactos
		\end{enumerate}
		
		
		\item \story{usuario}{poder buscar publicaciones en un foro}
		{poder buscar algo especifico dentro de un foro, y no tener que pasar por muchas publicaciones antes de encontrar lo que quiero}
		\begin{enumerate}
			\item Barra de búsqueda general que se pueda aplicar a foro, por ejemplo \say{$<$Nombre Foro$>$ + $<$contenido publicación$>$}
		\end{enumerate}
		
		
		\item \story{usuario}{poder buscar subscribirme y desubscribirme facilmente a un foro}{así poder ajustar convenientemente los temas que veo}
		\begin{enumerate}
			\item Implementar botón en la página inicial del foro boton de subscripción
		\end{enumerate}
		
		
		\item \story{usuario}{poder ver los foros recientemente creados}{saber que esta surgiendo y quizas encontrar nuevos foros favoritos}
		\begin{enumerate}
			\item Implementar foros recientemente creados en la página de inicio y aquellos que tengan más visitas
		\end{enumerate}
		
		
		
		\item \story{administrador}{poder acceder a un dashboard con estadisticas generales de la aplicacion}{así poder hacer los cambios de estructuras pertinentes de ser necesario}
		\begin{enumerate}
			\item Recolectar información de todos los posts y publicaciones
			\item Calcular media vistas, y tipo de vistas por post o por publicaciones
		\end{enumerate}
		
		\item \story{administrador}{que existan moderadores}{así poder delegar el 
			trabajo de administrar los foros}
		\begin{enumerate}
			\item Niveles de usuario con distintas capacidades para hacer cosas
			\item Moderadores son de un foro en específico
		\end{enumerate}
		
		\item \story{administrador}{que un usuario pueda pasar a ser un moderador con un
			determinado puntaje}{asi evitar que todos puedan moderar el foro}
		\begin{enumerate}
			\item Solicitud de \say{upgrade} de status de usuarios
			\item Solicitud no es aceptada inmediatamente
			\item Llega mail a admin o moderador con la solicitud
		\end{enumerate}
		
		\item \story{administrador}{que se asigne un puntaje por crear una publicación en un 	
			foro}{que así los usuarios puedan postular a \say{mejores cargos}}
		\begin{enumerate}
			\item Cada usuario tiene un puntaje total
			\item Diferentes acciones (upvotes) le dan un puntaje al creador del comentario, y otros quitan (downvote)	
		\end{enumerate}
		
		\item \story{administrador/moderador}{poder aceptar o rechazar a un postulante de 	
			mi cargo}{tener mejor control y un chequeo previo de a quién se le va a dar el cargo}
		\begin{enumerate}
			\item Generación de correo electrónico con la solicitud y con la respuesta a esta
			\item Sólo se puede postular con determinado puntaje en adelante
			\item Revisar su participación en el foro
		\end{enumerate}
		
		\item \story{administrador/moderador}{poder responder a la postulación hecha}{así 
			dar feedback}
		\begin{enumerate}
			\item Cuando se genere la solicitud le llega una notificación y un correo al administrador/moderador, ya que es de mayor relevancia
		\end{enumerate}
		
		\item \story{administrador}{poder crear y eliminar foros}{así poder evitar que el resto lo haga y como consecuencia exista un exceso de estos}
		\begin{enumerate}
			\item Administradores y moderadores tienen permisos especiales
			\item Estos permisos se chequean cuando se trata de hacer click en las opciones de foro
		\end{enumerate}
		
		\item \story{administrador/dueño}{que cuando el comentario de un usuario sea eliminado disminuya su puntaje}{así poder \say{castigar} comentarios inapropiados}
		\begin{enumerate}
			\item Luego de hacer click se manda la información al servidor para que se realicen los cambios
		\end{enumerate}
		
		\item \story{visita}{poder ver los foros ya existentes}{asi poder ver los temas que me interesan sin tener que registrarme en la página}
		\begin{enumerate}
			\item Cualquier persona puede acceder a la página web, por lo que el Log-In no puede impedir la entrada
			\item Sólo se puede interactuar con los demas si se está registrado
		\end{enumerate}				
		
		\item \story{administrador}{que el moderador tenga un puntaje}{así poder decidir quién puede ser administrador y quién no tras una postulación}
		\begin{enumerate}
			\item En otras palabras, el moderador también es un usuario, por lo que posee puntaje y debe postular para subir de \say{cargo}
		\end{enumerate}
		
		\item \story{usuario}{poder recomendar una idea de foro a un administrador}{para 	que esta idea sea evaluada y posteriormente tener feedback de este foro 					hipotético}
		\begin{enumerate}
			\item Mail con sugerencias que le llega a los administradores y luego deben aprobar o rechazar
			\item Se debe implementar feedback luego del resultado y notificando al usuario que hizo la sugerencia
		\end{enumerate}			
		
	\end{enumerate}
	
	
	
	
\end{document}